
% Vidya: This is a repeat from the intro, so commented out.

%Mobile users often determine that finding information with mobile devices is more difficult than with desktop machines, and have adapted their use of mobile devices accordingly.
%Cui and Roto~\cite{Cui:2008} and Church and Oliver~\cite{Church:2011} found that mobile users effectively limit their expectations by posing either fairly specific queries or extremely broad ones. Browsing can easily satisfy open broad queries, while search can quickly answer very specific queries. Lee \textit{et al.}~\cite{Lee:2012} point out that human queries are often ill-formed and refined iteratively on the basis of intermediate query results. In a typical case, a user knows what they seek, but does not know the keyword that would retrieve it. Some search engines offer partial solutions by showing auto-completed keyword sets in real time (e.g. Bing's search suggestions), or by displaying related facts from a database (e.g. Google's Knowledge Graph). Even so, users are often forced to perform several searches to help them find the appropriate keywords for their search.

\textit{GraphTiles} exploits structured data sources to facilitate information discovery, and has similarities to faceted search and various category-based interfaces. There are several systems designed to support faceted navigation, allowing users to explore a collection of information by applying multiple classification filters. FaThumb supports navigation of a hierarchical information space by incremental text entry and attribute based filtering using a numeric keypad~\cite{Karlson:2006}. While text entry is fastest if one knows the specific information, facet navigation is faster when one only knows the attributes of that information. The MuZeeker application supports category based filtering to refine search by category selection rather than typing additional text~\cite{Larsen:2010}. The system uses contextual information from the search results to relate individual search results to external resources such as YouTube videos. mSpace Mobile employs fish-eyed multi-panes, where each pane returns information for a specific facet~\cite{mspace}.


One way of thinking about \textit{GraphTiles} is that it exploits knowledge of information locality to improve search. Similarly, other mobile search tools often take advantage of user context such as location and time to provide a localized experience. Lymberopoulos \textit{et al.} apply a data-driven approach where a local search model at different levels of location granularity (\textit{e.g.} city, state, country) are combined together to improve click prediction accuracy in the search results \cite{Lymberopoulos:2011}. \textit{FindAll} is a local mobile search engine that lets users search and retrieve web pages, even in the absence of connectivity. The premise for their work is that mobile users often search for web pages that they have previously visited, known as re-finding. \textit{FindAll} estimates the benefits of local search by learning the re-finding behavior of users~\cite{Balasubramanian:2012}. \textit{Hapori}, a local mobile search tool,  not only takes into account location in the search query but richer context such as the time, weather and the activity of the user~\cite{Lane:2010}. Amini \textit{et al.} present Trajectory-Aware Search (TAS) that predicts the user's destination based on location data from the current trip and shows search results near the predicted location~\cite{Amini:2012}. \textit{SocialSearchBrowser} incorporates social networking capabilities with key mobile contexts to improve the search and information discovery experience of mobile users \cite{Church:2010}.

\textit{GraphTiles} is essentially a visualization of and search interface for the local entity-relationship graph. There has been little work specifically addressing mobile visualization ~\cite{RefWorks:658}, and to our knowledge, no work on mobile visualization for search. Karstens~\cite{RefWorks:908} proposes node-link diagrams of hierarchies arranged around a rectangle to make efficient use of display space. He displayed nearly 1000 nodes, each represented by a very small circle. Hao and Zhang ~\cite{RefWorks:906} propose a space-filling sunburst display of hierarchies. Their larger nodes are easier to interact with, but their graphs are much smaller. Pattath \textit{et al.}  \cite{RefWorks:896} visualize general graphs numbering just a few dozen nodes using node-link diagrams. Finally, in work most closely related to our own, Da Lozzo \textit{et al.}~\cite{springerlink:10.1007/978-3-642-18469-7-14} use node-link diagrams centered around a specific node, again with very small nodes. To recognize mobile constraints, \textit{GraphTiles} limits visualization to a graph neighborhood as do Da Lozzo \textit{et al.}, but like Hao and Zhang, it displays many links implicitly.
