Mobile search is becoming a prevalent everyday activity for millions of users to search, locate and discover information on-the-go.  Cui and Roto \cite{Cui:2008} and Church and Oliver \cite{Church:2011} found that mobile users effectively limit their expectations by posing either fairly specific \textit{direct} queries or extremely broad ones. Fact-finding, where the user searches for a specific piece of information, and browsing more open subject-based exploration, are examples of these kinds of tasks. 

However, mobile users may have difficulty articulating a precise query due to the lack of knowledge about the content and the limited query capability of the interface \cite{Kamvar:2009}. This can result in the users having to reformulate their queries several times on the basis of intermediate query results. For example, a user may have in mind a specific actor she
wishes to find on her mobile device. Her approach to search is influenced by whether she remembers the name of the actor
(in which case she would directly search by name) or she might instead search by some combination of movie and actor to find whom she's looking for.  This would lead to several iterations of search to finally find the answer. This``no man's land" of \textit{imprecise} search that is neither very open nor very specific remains unsupported. Lee \textit{et al.} \cite{Lee:2012} points out that these imprecise queries are more common than we might think. In a typical case, a user knows what they seek, but does not know the keyword that would retrieve it.

Leveraging the relationships between entities in search, whether searching for an actor in a graph of movies to cast in IMDb~\cite{imdb}, songs to artists in Pandora~\cite{pandora} or a network of friends in Facebook~\cite{Facebook}, not only helps in structuring search results, but as a tool to guide users in reformulating queries that are typical of imprecise search. This approach not only reduces cognitive load through recognition, but allows users to navigate search results by attribute values rather than by keyword alone \cite{Hearst:2002}.

Mobile search interfaces offer little more than traditional desktop search, with most using simple adaptations of their web interfaces to meet the display characteristics of
mobile handsets. Yet, the primary method of web navigation remains hyperlinks \cite{206540}.  These are well-suited for depth-first traversal, where a user first selects a link on the current page, which in turn loads a new page. This process repeats until the user finds the needed information or the current search path is abandoned returning the user to the initial search. Some search engines offer partial solutions by showing auto-completed keyword sets in real time (e.g. Bing's search suggestions), or by displaying related facts from a database (e.g. Google's Knowledge Graph). Even so, users are often forced to perform several searches to help them find the appropriate keywords for their search.


In this paper, we present \textit{GraphTiles}, a visual search interface for mobile devices, to help users perform imprecise queries. The interface leverages the neighborhood of entities and their relationships in a content graph, displaying an incomplete portion of a local link neighborhood: a thumbnail of the current page alone in the left column, some pages one link away in the middle column, and other pages two links away in the right column. To see the complete neighborhood, users can scroll the central and right columns vertically.

While this layout implies many links, it does not indicate exactly where the links are between the second and third columns. Users can reveal these locations by selecting a page thumbnail from these columns, triggering an \textit{interactive reordering} that highlights pages linked to the selection and places them onscreen or nearly so. Users can restore the original (better known) order by deselecting the page.


