According to recent reports, mobile search will soon surpass desktop search as measured by both queries and ad revenue \cite{MobileQueries}\cite{MobileRevenue}. Despite this growing importance, Cui and Roto \cite{Cui:2008} and Church and Oliver \cite{Church:2011} find that current mobile search interfaces lead users to seek only information that is fairly specific (e.g. fact-finding) or quite general (e.g. browsing). Both types of information can be retrieved by short, easily input queries.

Ideally mobile users should not have to limit themselves in this way, and indeed often they do not, either because they misjudge their ability to do so, or because of their pressing need for the information. Then their lack of knowledge about the information they seek and the limited query capability of the mobile interface \cite{Kamvar:2009} can force them to reformulate their queries several times on the basis of intermediate query results. For example, a user may seek a specific actor. If she cannot remember the name of the actor (in which case she would directly search by name) or she might instead search for an actor who worked with the actor they seek. 

As Lee \textit{et al.} \cite{Lee:2012} points out, this``no man's land" of \textit{imprecise} search (neither very general nor very specific) is more common than we might think, and some search engines have begun offering partial solutions. Search suggestions offer to complete keyword sets automatically in real time, helping users form better queries. Google's Knowledge Graph displays related facts from databases, improving the content and structure of search results. Yet neither solution is complete.

We believe that by leveraging the relationships between entities in search, whether searching for an actor in a graph of movies to cast in IMDb~\cite{imdb}, songs to artists in Pandora~\cite{pandora} or a network of friends in Facebook~\cite{Facebook}, not only helps in structuring search results, but as a tool to guide users in reformulating queries that are typical of imprecise search. This approach not only reduces cognitive load through recognition, but allows users to navigate search results by attribute values rather than by keyword alone \cite{Hearst:2002}.

In this paper, we present \textit{GraphTiles}, a visual search interface for mobile devices, to help users perform imprecise queries. The interface leverages the neighborhood of entities and their relationships in a content graph, displaying an incomplete portion of a local link neighborhood: a thumbnail of the current page alone in the left column, some pages one link away in the middle column, and other pages two links away in the right column. To see the complete neighborhood, users can scroll the central and right columns vertically.

While this layout implies many links, it does not indicate exactly where the links are between the second and third columns. Users can reveal these locations by selecting a page thumbnail from these columns, triggering an \textit{interactive reordering} that highlights pages linked to the selection and places them onscreen or nearly so. Users can restore the original (better known) order by deselecting the page.


