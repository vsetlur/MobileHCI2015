% $Id: template.tex 11 2007-04-03 22:25:53Z jpeltier $

%\documentclass{vgtc}                          % final (conference style)
\documentclass[review]{vgtc}                 % review
%\documentclass[widereview]{vgtc}             % wide-spaced review
%\documentclass[preprint]{vgtc}               % preprint
%\documentclass[electronic]{vgtc}             % electronic version

%% Uncomment one of the lines above depending on where your paper is
%% in the conference process. ``review'' and ``widereview'' are for review
%% submission, ``preprint'' is for pre-publication, and the final version
%% doesn't use a specific qualifier. Further, ``electronic'' includes
%% hyperreferences for more convenient online viewing.

%% Please use one of the ``review'' options in combination with the
%% assigned online id (see below) ONLY if your paper uses a double blind
%% review process. Some conferences, like IEEE Vis and InfoVis, have NOT
%% in the past.

%% Figures should be in CMYK or Grey scale format, otherwise, colour 
%% shifting may occur during the printing process.

%% These three lines bring in essential packages: ``mathptmx'' for Type 1 
%% typefaces, ``graphicx'' for inclusion of EPS figures. and ``times''
%% for proper handling of the times font family.
\let\ifpdf\relax
\usepackage{mathptmx}
\usepackage{graphicx}
\usepackage{times}

%% We encourage the use of mathptmx for consistent usage of times font
%% throughout the proceedings. However, if you encounter conflicts
%% with other math-related packages, you may want to disable it.

%% If you are submitting a paper to a conference for review with a double
%% blind reviewing process, please replace the value ``0'' below with your
%% OnlineID. Otherwise, you may safely leave it at ``0''.
\onlineid{108}

%% declare the category of your paper, only shown in review mode
\vgtccategory{Workshop}

%% allow for this line if you want the electronic option to work properly
\vgtcinsertpkg

%% In preprint mode you may define your own headline.
%\preprinttext{To appear in an IEEE VGTC sponsored conference.}

%% Paper title.

\title{Research Challenges and Opportunities in Mobile Visualization}

%% This is how authors are specified in the conference style

%% Author and Affiliation (single author).
%%\author{Roy G. Biv\thanks{e-mail: roy.g.biv@aol.com}}
%%\affiliation{\scriptsize Allied Widgets Research}

%% Author and Affiliation (multiple authors with single affiliations).
%%\author{Roy G. Biv\thanks{e-mail: roy.g.biv@aol.com} %
%%\and Ed Grimley\thanks{e-mail:ed.grimley@aol.com} %
%%\and Martha Stewart\thanks{e-mail:martha.stewart@marthastewart.com}}
%%\affiliation{\scriptsize Martha Stewart Enterprises \\ Microsoft Research}

%% Author and Affiliation (multiple authors with multiple affiliations)
\author{Benjamin Watson\thanks{e-mail: bwatson@ncsu.edu}\\ %
        \scriptsize North Carolina State University \\
        watson.csc.ncsu.edu %
\and Vidya Setlur\thanks{e-mail:vsetlur@tableau.com}\\ %
     \scriptsize Tableau Software \\
        vidyasetlur.com} %}}

%% A teaser figure can be included as follows, but is not recommended since
%% the space is now taken up by a full width abstract.
%\teaser{
%  \includegraphics[width=1.5in]{sample.eps}
%  \caption{Lookit! Lookit!}
%}

%% Abstract section.
\abstract{As people shift to mobile devices for their daily information consumption, we are faced with the challenge of designing new interfaces and interactions that work with these smaller screens. For analysis and presentation of data visualizations in general, resizing is particularly critical in the context of a dashboard with limited real-estate, and/or when visualizations created on one display must then rendered on a different display. With the constraints of the form factor and touch-screen modalities that are often prevalent, better interaction techniques need to be explored. In addition, mobiles contain active sensors such as a camera and GPS making them producers of vast amounts of data. On the contrary, with increased computation power, they are capable of consuming large amounts of data as well. Determining the balance between consumption and production of big data is another area of research for the visualization community.

This workshop will be dedicated to the identification, organization and discussion of the challenges and research opportunities in mobile visualization, and the organization of a new research community centered around them.

} % end of abstract

%% ACM Computing Classification System (CCS). 
%% See <http://www.acm.org/class/1998/> for details.
%% The ``\CCScat'' command takes four arguments.
%
%\CCScatlist{ 
%  \CCScat{K.6.1}{Management of Computing and Information Systems}%
%{Project and People Management}{Life Cycle};
%  \CCScat{K.7.m}{The Computing Profession}{Miscellaneous}{Ethics}
%}

%% Copyright space is enabled by default as required by guidelines.
%% It is disabled by the 'review' option or via the following command:
% \nocopyrightspace

%%%%%%%%%%%%%%%%%%%%%%%%%%%%%%%%%%%%%%%%%%%%%%%%%%%%%%%%%%%%%%%%
%%%%%%%%%%%%%%%%%%%%%% START OF THE PAPER %%%%%%%%%%%%%%%%%%%%%%
%%%%%%%%%%%%%%%%%%%%%%%%%%%%%%%%%%%%%%%%%%%%%%%%%%%%%%%%%%%%%%%%%

\begin{document}

%% The ``\maketitle'' command must be the first command after the
%% ``\begin{document}'' command. It prepares and prints the title block.



\maketitle

\section{Organizers}
%Benjamin Watson


\subsection{Benjamin Watson}
Associate Professor\\
North Carolina State University, Department Computer Science\\
email: bwatson@ncsu.edu, tel: +1 919 513 0325\\
website: watson.csc.ncsu.edu\\

Watson`s Visual Experience Lab studies visual technologies that move people: how digitally created imagery affects human emotion, thinking and behavior. His work spans computer graphics, human-computer interfaces, visualization, psychology and design. Much of his work has migrated to the mobile platform, as the most pervasive of visual interfaces. Watson co-chaired the Graphics Interface 2001, IEEE Virtual Reality 2004 and ACM Interactive 3D Graphics and Games (I3D) 2006 conferences, and was co-program chair of I3D 2007. He served several years as Conference Chair on the VGTC. Watson is an ACM and senior IEEE member. He earned his doctorate at the Graphics, Visualization and Usability Center of the Georgia Institute of Technology. 

\subsubsection{Related publications:}
\begingroup
\renewcommand{\section}[2]{}%
\begin{thebibliography}{4} 

\bibitem{BaeWatson2014} J. Bae and B.A. Watson. Reinforcing Visual Grouping Cues to Communicate Complex Informational Structure. \emph{IEEE Transactions on Visualization and Computer Graphics}, 20, 12, pp. 1973--1982, 2014.\\

\bibitem{BaeWatsom2011} J. Bae and B.A. Watson. Developing and evaluating Quilts for the depiction of large layered graphs. \emph{IEEE Transactions on Visualization and Computer Graphics}, 17, 12, pp. 2268-2275, 2011.\\

\bibitem{feketewatson2010} J.-D. Fekete, P. Dragicevic, A. Bezerianos, J. Bae and B.A. Watson. GeneaQuilts: a system for exploring large genealogies. \emph{IEEE Transactions on Visualization and Computer Graphics}, 16, 6, 1073-1081, 2010.\\

\bibitem{albrechtwatson2005} C. Albrecht-Buehler, B.A. Watson and D.A. Shamma. Visualizing live text streams using motion and temporal pooling. \emph{IEEE Computer Graphics and Applications, Special issue on Smart Depiction for Visual Communication.}, 25, 3, pp. 52--59, 2005.

\end{thebibliography}
\endgroup







\subsection{Vidya Setlur}
Research Scientist\\
Tableau Software\\
email: vsetlur@tableau.com, tel: +1 650 796 6097\\
website: http://vidyasetlur.com\\

Vidya Setlur is a research scientist at Tableau Software. For several years before that, she was a principal research scientist at Nokia Research Center. Her research interest lies at the intersection of natural language processing and computer graphics. Her work involves new rendering algorithms and mobile interfaces, particularly in the area of iconography, visualization and content retargeting for small displays. In addition to industry research, she enjoys teaching and giving courses at various academic and conference settings, including an appointment with Carnegie Mellon University (Silicon Valley campus) as an adjunct professor. She earned her doctorate in Computer Graphics at Northwestern University in 2005.

\subsubsection{Related publications:}
\begingroup
\renewcommand{\section}[2]{}%
\begin{thebibliography}{5} 

\bibitem{SetlurGooch2011} V. Setlur, S. Rossoff and B. Gooch. Wish I Hadn't Clicked That: Context Based Icons for Mobile Web Navigation and Directed Search Tasks. \emph{Intelligent User Interfaces (IUI)}, 2011.\\

\bibitem{SohnSetlur2010} T. Sohn, V. Setlur, K. Mori, J. Kaye, H. Horii, A. Battestini, R. Ballagas, C. Paretti and M. Spasojevic. Addressing Mobile Information Overload in the Universal Inbox through Lenses. \emph{Mobile Human Computer Interaction (MobileHCI)}, 2010.\\
 
\bibitem{Battestini2010} A. Battestini, V. Setlur, T. Sohn. A Large Scale Study of Text Messaging Use. \emph{Mobile Human Computer Interaction (MobileHCI)}, 2010.\\

\bibitem{Setlur2007} V. Setlur, M. Neinhaus, T. Lechner and B. Gooch. Retargeting Images for Preserving Image Saliency. \emph{IEEE Computer Graphics and Applications}, 2007. \\

\bibitem{Setlur2005} V. Setlur, C. Albrecht-Buehler, A. Gooch, S. Rossoff and B. Gooch. Semanticons: Semantic Based File Icons. \emph{Eurographics}, 2005.\\
\end{thebibliography}
\endgroup
The two organizers most recently have a paper, GraphTiles: A Visual Interface for Supporting Browsing and Imprecise Mobile Search, conditionally accepted at Mobile HCI 2015.

\section{Goals and Scope}
As mobile devices become the dominant form of computing, the field of visualization needs to consider the challenges and research opportunities mobile computing presents if it is to continue growing in relevance. This workshop will be dedicated to the identification, organization and discussion of those topics, and the organization of a new research community centered around them.
Such topics possibly include: \\
\begin{itemize}
\item \textbf{Scenarios for mobile visualization:} Given mobile usage constraints such as limited input bandwidth, display size, glare, and distraction, what are the valid use cases for mobile visualization? Possible use cases might include:
\begin{itemize}
\item Visualization dashboards organize and present collections of visualizations on a single view for easy presentation and monitoring. Mobility provides an added advantage to be able to view and interact with this information and receive alerts on-the-go. How could mobile visualization improve such applications?

\item From their very inception, a primary use case for mobile devices has been communication. How can visualization improve communication, and how can the combination of mobility and visualization help disseminate knowledge?

\item Both the mobile and visualization research communities have become increasingly interested in persuasive applications. How can these two technologies combine to deliver messages more persuasively?
\end{itemize}

\item \textbf{Rendering and interaction:} How must visualizations change to accommodate mobile constraints? 

\begin{itemize}
\item Although mobile displays often contain more pixels than traditional displays, they are much smaller in size. How can visualizations be fit onto mobile displays while remaining useful for visual analysis?

\item Are there alternative visualizations that could be more effective on a mobile? For example, could heatmaps be preferable to circumvent the problem of overplotting with scatterplots, particularly on a small screen?

\item Unlike most traditional settings, mobile visualization requires interaction with fingers directly on the display surface. How can visualization accommodate touch interaction? Could interaction techniques such as scroll, pan and zoom, overview + detail be effectively applied here?
\end{itemize}

\item \textbf{Mobiles and big data:} What role will mobile devices play in meeting the challenge of big data? 
\begin{itemize}
\item Mobile sensing creates new data that are currently only poorly understood, how can visualization help change this? These new data include: 
traces of human movement, communication and work
data types such as text, sound, imagery and video
crowdsourced, real time, big data flows 
\item Certainly mobile crowdsourcing can generate big data, can it also play a role in understanding that data?  Could visualization become part of a mobile feedback loop, with sensing to the cloud and visualization from it?
\end{itemize}

\item \textbf{Authoring and consuming visualizations on mobile:} Mobiles have changed how information is discovered and consumed, greatly increasing targeted, monitoring and grazing information seeking. Much of today�s digital content is also authored on mobile devices. For example, over 350 million photos are uploaded to Facebook each day. 
\begin{itemize}
\item How should visualization change to meet the needs of mobile information consumption? 
\item How might visualizations be authored on mobiles?
\item Might visualization be collaboratively consumed and created on mobile devices?
\end{itemize}
\end{itemize}

\section{Activities}
In the first half of the workshop, participants will present short position or research papers. Each presentation will be followed by discussion, and the issues they raise. 
The second half of the workshop will be dedicated to the production of a research agenda. With participants� help, we will identify several broad challenges, and organize participants into groups focused on them. Each group will then discuss that challenge in detail, and report back to participants as a whole. We will record these reports.
After the event, we will perpetuate the workshop�s agenda and impact by creating a website hosting papers, a final workshop report, and listing participants. We will also create a wiki containing a bibliography of related research, and an email list to help facilitate among participants. If participant interest merits it, we will also assist participants in organizing a subsequent event. 

\section{Length}
Full day (but could be abridged to a half day format with shorter paper presentations and subsequent discussion of a smaller set of topics.)

\section{Organization}
Watson and Setlur will serve as organizing workshop chairs, and help review submissions. Should the workshop be approved, we will also recruit a small program committee to help with the reviewing process, drawing from both the visualization and mobile HCI communities.
We anticipate selecting roughly 8 papers for presentation. Papers will be selected by consensus of the program committee. Should this process not fill the program, we will select from remaining papers by majority vote. We would also like to accommodate 10-20 additional attendees to participate in discussion. 
Given our workshop format with intensive discussion, we would like the meeting room set up with round tables (though chairs alone would do in a pinch). However to accommodate presentations, we would also like to have a podium and projector, much like a talk during a banquet.

\section{Results and Impact}
The goal of this workshop is to ensure that visualization remains useful on emerging mobile platforms. The workshop will move toward this goal by:
\begin{itemize}
\item Building a lasting community of interested researchers with introductions during the event and social networking after. 

\item Help identify unique opportunities for exploring interesting and relevant research questions for visual analysis on mobiles.

\item Defining and promoting a research agenda in mobile visualization.
\end{itemize}

\end{document}
