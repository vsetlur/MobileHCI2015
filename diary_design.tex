
In order to learn how people perform search with mobile devices in everyday use, we ran a two-week diary study which collected individual search information. We provided each participant with a diary booklet to keep a history of their online searches. They were allowed to change keywords if necessary and redo the search.

\subsubsection{Design}
The booklet contained 25 pages and each page included the questions listed below. If participants were not able to find an appropriate answer, they provided an explanation. We asked the participants to write down these details as soon as possible after they performed a search. We informed the participants that they should only divulge information that they were comfortable haring, and we may publish anonymized quotes from their diaries. 

These are the questions on each page of the booklet participants answered as soon as they performed a search.
\begin{tight_enumerate}
            \item Date
            \item Time
            \item Duration of search task 
            \item What app or website did you access
            \item What were you searching for?
            \item Did you find what you were searching for at all? YES/NO
            \item If you did find your information, please continue by filling in the blanks with numbers: I performed \_ searches to find my information. I followed \_ links after leaving the search results page.
            \item Rate the difficulty of finding your information from 1--5 with 5 being very difficult. Add text to explain if you like.
\end{tight_enumerate}


\subsection{Method}

\subsubsection{Participants}
32 people (21 college students, age 18--62, 17 male, 15 female) participated in the experiment. All had normal or corrected-normal vision. They were required to have a mobile device capable of search, and to be regular users of that functionality. 

\subsubsection{Apparatus} 
The experiment was performed on individual's smartphones. 10 participants had iOS, 11 had Android, and one participant had a Windows phone. Others did not report their system.

\subsubsection{Procedure} 

We gathered participants via word-of-mouth, email, and flyers. We first obtained informed consent from the participants, and gave them written instructions explaining the questions in the booklet. 

Participants were asked to provide their name, gender, email, type of mobile device, and whether they wanted a gift card or credits for class. We asked them to record two searches per day in order to fill out a 25-page booklet over the two week period. We met each participant after a week in order to check their diaries and data, answer any questions, and help them improve their feedback. During the meeting, we audio-recorded the dialog to archive quotes and feedback. After the second week, we collected the booklets. We then either paid participants \$9 or gave them three cupcakes, or they earned class credit. Each participant was assigned a unique ID to maintain their anonymity. It was used to send out reminders for the weekly meetings. If a participant completed a booklet before two weeks were over, we gave them a new one to fill out. We informed participants that they could terminate the experiment at any time.
