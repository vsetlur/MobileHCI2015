% Here is my plan for this section: 
% A) characterize the problem
% 1) define imprecise search as searches that require multiple queries (difficulty forming query) or multiple link following (difficulty navigating results) 
% 2) define difficult search as failed, rated difficult or taking a long time 

%Imprecise means 2 or more queries or 3 or more links clicked. 
%Too hard means failed, difficulty rating 4 or 5, or more than 2 minutes work. 

% 3) find how often imprecise happens, how often difficult happens, and whether difficult happens more often with imprecise 
% B) characterize a potential solution 
% 1) define structured data queries as search of known structured datasets or queries that could be answered by same 
% 2) find how often structured data queries happen, how often they intersect with imprecise and imprecise-difficult queries. 
% C) test hypotheses that 
% 1) imprecise queries are common enough todesign solutions for
% 2) imprecise queries are often difficult 
% 3) a solution for structured data queries would solve a significant portion of these 


\begin{table*}[ht]
\begin{center}
    \begin{tabular}{ | c | c | c | l |}
    \hline
    Category &  Number of Searches & Percentage & Query Examples \\ \hline
   \multirow{3}{*}{Precise and Easy} & & &  ``lakes around raleigh'' \\
    					   & 425 & 49 & ``data mining companies in the US''\\ 				
					   & &  &  ``lenovo a580 review''  \\ \hline 
					   
   \multirow{3}{*}{Precise and Difficult} & & & ``Where can I buy beautiful ruins at lowest price?'"\\
    					   & 61 & 7 & ``home remedy for cat diarrhea''\\ 				
					   & &  & ``how to transfer when taking a grey hound''  \\ \hline 

   \multirow{2}{*}{ Imprecise/General and Easy} & & & ``labrador dog breeder'"\\
    					   & 174 & 20 & ``flights to west coast''\\
					   & & & ``halloween costumes "\\ \hline 

   \multirow{2}{*}{ Imprecise/General and Difficult} & & & ``salmon recipes"\\
    					   & 208 & 24 & ``name of movie with actors chang and bling''\\
					   & & & ``bathroom vanity mirror, bathroom mirror"\\ \hline 

   
 \end{tabular}
\end{center}
\label{searchcategories}
\caption{Four categories of mobile searches in the diary study, their frequency of occurrence and examples.}
\end{table*}

We wanted to understand how often people perform \textit{imprecise} searches in regular use, and how searches influence difficulty. To capture mobile users outside of the lab, we opted for a two-week diary study, in which participants record their own behavior in their paper diaries, a technique similar to what is described in \cite{Sohn:2008}. 

 
Imprecise searches can be characterized by at least one of two properties \cite{Lee:2012}:
\begin{tight_enumerate}
\item Users iteratively refine multiple queries to find relevant information due to difficulty formulating an exact query. 
\item Users have difficulty navigating through their search results to find the answer they are looking for, leading to multiple link following.
\end{tight_enumerate}

Accordingly we formulated the following definition of \textit{imprecise} search, as measured by our diary study: \textit{more than one query was required, or three or more links in the result list were followed}. However, we found one ambiguous case: one query and at least three followed links might be imprecise search, with a user arriving directly at a confusing set of search results and hunting around; or they might be general browsing search, with a user quickly finding a broad swath of interesting information, and slowly exploring it. Disambiguation might require knowing how rapidly users followed their links. Unfortunately, this sort of timing information is not reliable in diary studies. We therefore settled for grouping imprecise with general searches in our design. 

We now describe the participant profile, web diary tool, and study procedure of our diary study.

\subsection{Participants}
We recruited $32$ participants ($21$ college students, $8$ software professionals, $2$ office secretaries, and $1$ school teacher) through online mailing lists and flyers. Their ages ranged between $18$ and $62$, with $17$ being male and $15$ female. All had normal or corrected-normal vision. They were required to have a mobile device capable of search, and to be regular users of that functionality. $14$ participants had iOS, $11$ had Android, and $7$ had Windows phones. 

\subsection{Procedure}
We provided each participant with a diary booklet to keep a history of their online searches.  We asked them to record at least two searches per day in order to fill out a 25-page booklet over the two week period. We met each participant after a week in order to check their diaries and data, answer any questions, and help them improve their feedback. During the meeting, we audio-recorded the dialog to archive quotes and feedback. After the second week, we collected the booklets. Participants were either compensated \$9 or earned class credit. Each participant was assigned a unique ID to maintain their anonymity. If a participant completed a booklet before two weeks were over, we gave them a new one to fill out. We informed participants that they could terminate the experiment at any time, and that they should only divulge information that they were comfortable sharing. We also mentioned that we may publish anonymized quotes from their diaries. 

The booklet contained 25 pages and each page included the questions listed below. If participants were not able to find an appropriate answer, they provided an explanation. We asked the participants to write down these details as soon as possible after they performed a search. 

These were the questions on each page of the diary that participants answered as soon as they performed a search.
\begin{tight_enumerate}
            \item Date
            \item Time
            \item Duration of search task 
            \item What app or website did you access
            \item What were you searching for?
            \item Did you find what you were searching for at all? YES/NO
            \item If you did find your information, please continue by filling in the blanks with numbers: I performed \_ searches to find my information. I followed \_ links after leaving the search results page.
            \item Rate the difficulty of finding your information from 1--5 with 5 being very difficult. Add text to explain if you like.
\end{tight_enumerate}


