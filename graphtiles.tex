\documentclass{sigchi}

% Use this command to override the default ACM copyright statement (e.g. for preprints). 
% Consult the conference website for the camera-ready copyright statement.


%% EXAMPLE BEGIN -- HOW TO OVERRIDE THE DEFAULT COPYRIGHT STRIP -- (July 22, 2013 - Paul Baumann)
% \toappear{Permission to make digital or hard copies of all or part of this work for personal or classroom use is 	granted without fee provided that copies are not made or distributed for profit or commercial advantage and that copies bear this notice and the full citation on the first page. Copyrights for components of this work owned by others than ACM must be honored. Abstracting with credit is permitted. To copy otherwise, or republish, to post on servers or to redistribute to lists, requires prior specific permission and/or a fee. Request permissions from permissions@acm.org. \\
% {\emph{CHI'14}}, April 26--May 1, 2014, Toronto, Canada. \\
% Copyright \copyright~2014 ACM ISBN/14/04...\$15.00. \\
% DOI string from ACM form confirmation}
%% EXAMPLE END -- HOW TO OVERRIDE THE DEFAULT COPYRIGHT STRIP -- (July 22, 2013 - Paul Baumann)


% Arabic page numbers for submission. 
% Remove this line to eliminate page numbers for the camera ready copy
\pagenumbering{arabic}


% Load basic packages
\usepackage{balance}  % to better equalize the last page
\usepackage{graphics} % for EPS, load graphicx instead
\usepackage{graphicx}
\usepackage{times}    % comment if you want LaTeX's default font
\usepackage{url}      % llt: nicely formatted URLs
\usepackage[T1]{fontenc} 
\usepackage{subfigure}
\usepackage{balance}
\usepackage{url}
\usepackage{enumitem}
\usepackage{times}
%\usepackage{slashbox}
\usepackage{multirow}
\usepackage{color}
\usepackage{array}
\usepackage[table]{xcolor}
% http://ctan.org/pkg/xcolor   %{colortbl}
%\usepackage{slashbox}
\usepackage{colortbl}
\usepackage{color}
\definecolor{dark_green}{rgb}{0.2, 0.8, 0.0}
\definecolor{dark_yellow}{rgb}{0.9, 0.6, 0.1} %{0.6, 0.4, 0.2}
\usepackage{makecell}
\usepackage{tikz}


% llt: Define a global style for URLs, rather that the default one
\makeatletter
\def\url@leostyle{%
  \@ifundefined{selectfont}{\def\UrlFont{\sf}}{\def\UrlFont{\small\bf\ttfamily}}}
\makeatother
\urlstyle{leo}


% To make various LaTeX processors do the right thing with page size.
\def\pprw{8.5in}
\def\pprh{11in}
\special{papersize=\pprw,\pprh}
\setlength{\paperwidth}{\pprw}
\setlength{\paperheight}{\pprh}
\setlength{\pdfpagewidth}{\pprw}
\setlength{\pdfpageheight}{\pprh}

% Make sure hyperref comes last of your loaded packages, 
% to give it a fighting chance of not being over-written, 
% since its job is to redefine many LaTeX commands.
\usepackage[pdftex]{hyperref}
\hypersetup{
pdftitle={SIGCHI Conference Proceedings Format},
pdfauthor={LaTeX},
pdfkeywords={SIGCHI, proceedings, archival format},
bookmarksnumbered,
pdfstartview={FitH},
colorlinks,
citecolor=black,
filecolor=black,
linkcolor=black,
urlcolor=black,
breaklinks=true,
}

%\usepackage{ftnright}

% create a shortcut to typeset table headings
\newcommand\tabhead[1]{\small\textbf{#1}}
\newcommand{\vidya}[1]{\textcolor{red}{(vidya: #1)}}
\newenvironment{tight_itemize}{\begin{itemize} \itemsep
-2pt}{\end{itemize}}
\newenvironment{tight_enumerate}{\begin{enumerate} \itemsep
-2pt}{\end{enumerate}}


\toappear 

\teaser{
\centering
\includegraphics[width=7in]{images/teaser}
\caption{Comparing \textit{GraphTiles} with IMDb's mobile website. (a) and (b): The MPM(Movie-person-movie) \textit{QueryType}; (c) and (d): the PMP(Person-movie-person) \textit{QueryType}.}
\label{fig:teaser}
}



% End of preamble. Here it comes the document.
\begin{document}

\title{GraphTiles: A Visual Interface for Supporting \\Imprecise Mobile Search}
%
%\numberofauthors{3}
%\author{
%  \alignauthor 1st Author Name\\
%    \affaddr{Affiliation}\\
%    \affaddr{Address}\\
%    \email{e-mail address}\\
%    \affaddr{Optional phone number}
%  \alignauthor 2nd Author Name\\
%    \affaddr{Affiliation}\\
%    \affaddr{Address}\\
%    \email{e-mail address}\\
%    \affaddr{Optional phone number}    
%  \alignauthor 3rd Author Name\\
%    \affaddr{Affiliation}\\
%    \affaddr{Address}\\
%    \email{e-mail address}\\
%    \affaddr{Optional phone number}
%}


\maketitle

\toappear

\begin{abstract}
Mobiles are generating a rapidly increasing proportion of search queries, ranging from specific fact-finding to unstructured exploration. Yet, search interfaces have not changed significantly to accommodate mobile constraints. Mobile search can be particularly challenging when traversing and learning about data relationships, such as those described in IMDb~\cite{imdb} and LinkedIn~\cite{linkedin}. We examined the prevalence of these mobile search use cases in a two-week diary study. We hypothesized that the ability to view a link neighborhood around the search result could be quite helpful, particularly for more imprecise, open queries. For example, one user reported searching for the location where `The sisterhood of the traveling pants' was filmed. Although their search was successful, it was difficult. A visual overview of such relationships could have been very helpful. To provide such overviews, we designed \textit{GraphTiles}, a visual interface supporting mobile search. In an experimental evaluation, users found information more quickly with \textit{GraphTiles} than with a standard mobile site.
\end{abstract}
%I'd like to see a bit more discussion on imprecise search here. Also, I'll see if I can't find something on entity relationship searches from the diary study.

\keywords{
	 Mobile search, connected data, imprecise queries.
	%\textcolor{red}{Mandatory section to be included in your final version.}
}


\category{H.5.m.}{Information Interfaces and Presentation (e.g. HCI)}{Miscellaneous}

%See: \url{http://www.acm.org/about/class/1998/}
%for more information and the full list of ACM classifiers
%and descriptors. 
%\textcolor{red}{Mandatory section to be included in your
%final version. On the submission page only the classifiers'
%letter-number combination will need to be entered.}






\section{Introduction}

%The majority of new computing devices sold are mobile ~\cite{IDC1,IDC2}. 
Mobile search is becoming a prevalent everyday activity for millions of users to search, locate and discover information on-the-go \cite{Church:2011}. Most people use the mobile Internet for \textit{directed} searches, where the goal is to find information about a predetermined topic of interest \cite{1143118}. Fact-finding, where the user searches for a specific piece of information, and the more open subject-based exploration, where the user seeks basic understanding of a concept and related terms, are two examples of this kind of task. Much of this information contains relations and connections, such as IMDb~\cite{imdb} (movies to cast), Pandora~\cite{pandora} (songs to artists) and Facebook~\cite{Facebook} (friends to friends). 


Mobile search interfaces offer little more than traditional desktop search, with most using simple adaptations of their web interfaces to meet the display characteristics of
mobile handsets. Yet, the primary method of web navigation remains hyperlinks \cite{206540}.  These are well-suited for depth-first traversal, where a user first selects a link on the current page, which in turn loads a new page. This process repeats until the user finds the needed information or the current search path is abandoned returning the user to the initial search. Such web navigation techniques tend to encourage `web surfing,' when a user starts in search of some information, but is sidetracked by tangential links.

In this paper, we present \textit{GraphTiles}, a visual search interface for mobile devices, particularly for data in a relational graph format. The interface displays an incomplete portion of a local link neighborhood: a thumbnail of the current page alone in the left column, some pages one link away in the middle column, and other pages two links away in the right column. To see the complete neighborhood, users can scroll the central and right columns vertically.

While this layout implies many links, it does not indicate exactly where the links are between the second and third columns. Users can reveal these locations by selecting a page thumbnail from these columns, triggering an \textit{interactive reordering} that highlights pages linked to the selection and places them onscreen or nearly so. Users can restore the original (better known) order by deselecting the page.




\section{Contributions}
The main contributions of this paper are:
\begin{itemize}
\item In a two-week diary study, we learned that most of the difficulty mobile search users experienced occurred during imprecise or general (browsing) searches.  
\item We designed the `GraphTiles' system, supporting imprecise and general searches on mobile devices.
\item In a controlled experiment, demonstrated that users were able to perform imprecise searches more quickly with GraphTiles than with a standard mobile website.
\end{itemize}





\section{Related Work}

% Vidya: This is a repeat from the intro, so commented out.

%Mobile users often determine that finding information with mobile devices is more difficult than with desktop machines, and have adapted their use of mobile devices accordingly.
%Cui and Roto~\cite{Cui:2008} and Church and Oliver~\cite{Church:2011} found that mobile users effectively limit their expectations by posing either fairly specific queries or extremely broad ones. Browsing can easily satisfy open broad queries, while search can quickly answer very specific queries. Lee \textit{et al.}~\cite{Lee:2012} point out that human queries are often ill-formed and refined iteratively on the basis of intermediate query results. In a typical case, a user knows what they seek, but does not know the keyword that would retrieve it. Some search engines offer partial solutions by showing auto-completed keyword sets in real time (e.g. Bing's search suggestions), or by displaying related facts from a database (e.g. Google's Knowledge Graph). Even so, users are often forced to perform several searches to help them find the appropriate keywords for their search.

\textit{GraphTiles} exploits structured data sources to facilitate information discovery, and has similarities to faceted search and various category-based interfaces. There are several systems designed to support faceted navigation, allowing users to explore a collection of information by applying multiple classification filters. FaThumb supports navigation of a hierarchical information space by incremental text entry and attribute based filtering using a numeric keypad~\cite{Karlson:2006}. While text entry is fastest if one knows the specific information, facet navigation is faster when one only knows the attributes of that information. The MuZeeker application supports category based filtering to refine search by category selection rather than typing additional text~\cite{Larsen:2010}. The system uses contextual information from the search results to relate individual search results to external resources such as YouTube videos. mSpace Mobile employs fish-eyed multi-panes, where each pane returns information for a specific facet~\cite{mspace}.


One way of thinking about \textit{GraphTiles} is that it exploits knowledge of information locality to improve search. Similarly, other mobile search tools often take advantage of user context such as location and time to provide a localized experience. Lymberopoulos \textit{et al.} apply a data-driven approach where a local search model at different levels of location granularity (\textit{e.g.} city, state, country) are combined together to improve click prediction accuracy in the search results \cite{Lymberopoulos:2011}. \textit{FindAll} is a local mobile search engine that lets users search and retrieve web pages, even in the absence of connectivity. The premise for their work is that mobile users often search for web pages that they have previously visited, known as re-finding. \textit{FindAll} estimates the benefits of local search, by learning the re-finding behavior of users~\cite{Balasubramanian:2012}. \textit{Hapori}, a local mobile search tool,  not only takes into account location in the search query but richer context such as the time, weather and the activity of the user~\cite{Lane:2010}. Amini \textit{et al.} present Trajectory-Aware Search (TAS) that predicts the user's destination based on location data from the current trip and shows search results near the predicted location~\cite{Amini:2012}. \textit{SocialSearchBrowser} incorporates social networking capabilities with key mobile contexts to improve the search and information discovery experience of mobile users \cite{Church:2010}.

\textit{GraphTiles} is essentially a visualization of and search interface for the local entity-relationship graph. There has been little work specifically addressing mobile visualization ~\cite{RefWorks:658}, and to our knowledge, no work on mobile visualization for search. Karstens ~\cite{RefWorks:908} proposes node-link diagrams of hierarchies arranged around a rectangle to make efficient use of display space. He displayed nearly 1000 nodes, each represented with a very small circle. Hao and Zhang ~\cite{RefWorks:906} propose a space-filling sunburst display of hierarchies. Their larger nodes are easier to interact with, but their graphs are much smaller. Pattath \textit{et al.}  \cite{RefWorks:896} visualize general graphs numbering just a few dozen nodes using node-link diagrams. Finally, in work most closely related to our own, Da Lozzo \textit{et al.}\cite{springerlink:10.1007/978-3-642-18469-7-14} use node-link diagrams centered around a specific node, again with very small nodes. To recognize mobile constraints, \textit{GraphTiles} limits visualization to a graph neighborhood as do Da Lozzo \textit{et al.}, but like Hao and Zhang, it displays many links implicitly.



\section{Diary Study}
% Here is my plan for this section: 
% A) characterize the problem
% 1) define imprecise search as searches that require multiple queries (difficulty forming query) or multiple link following (difficulty navigating results) 
% 2) define difficult search as failed, rated difficult or taking a long time 

%Imprecise means 2 or more queries or 3 or more links clicked. 
%Too hard means failed, difficulty rating 4 or 5, or more than 2 minutes work. 

% 3) find how often imprecise happens, how often difficult happens, and whether difficult happens more often with imprecise 
% B) characterize a potential solution 
% 1) define structured data queries as search of known structured datasets or queries that could be answered by same 
% 2) find how often structured data queries happen, how often they intersect with imprecise and imprecise-difficult queries. 
% C) test hypotheses that 
% 1) imprecise queries are common enough todesign solutions for
% 2) imprecise queries are often difficult 
% 3) a solution for structured data queries would solve a significant portion of these 

We wanted to understand how often people perform \textit{imprecise} queries, and how much of an influence those queries have on search difficulty. Imprecise queries can be characterized by at least one of two properties \cite{Lee:2012}: 
\begin{tight_enumerate}
\item Users iteratively refine multiple queries to find relevant information due to difficulty formulating an exact query. 
\item Users have difficulty navigating through their search results to find the answer they are looking for, leading to multiple link following.
\end{tight_enumerate}

% Probably need a quote fix here.
Capturing typical mobile user behavior is particularly challenging because of the difficulty of following them through their daily lives. Diary studies let users "follow themselves". We ran a two-week diary study which collected individual search information \cite{Sohn:2008}. We now describe the participant profile, web diary tool, and study procedure.

% Looks like we're missing a brief description of your participants vidya — e.g. 21 college students and 11 professionals? Did we not know which OS your participants used? If not we should just remove the one sided detail.
\subsection{Participants}
We recruited 32 participants (21 college students, age 18--62, 17 male, 15 female) through online mailing lists and flyers. All had normal or corrected-normal vision. They were required to have a mobile device capable of search, and to be regular users of that functionality. 10 participants had iOS, 11 had Android, and one participant had a Windows phone. Others did not report their system.


\subsection{Procedure}
We provided each participant with a diary booklet to keep a history of their online searches.  We asked them to record at least two searches per day in order to fill out a 25-page booklet over the two week period. We met each participant after a week in order to check their diaries and data, answer any questions, and help them improve their feedback. During the meeting, we audio-recorded the dialog to archive quotes and feedback. After the second week, we collected the booklets. Participants were either compensated \$9 or earned class credit. Each participant was assigned a unique ID to maintain their anonymity. If a participant completed a booklet before two weeks were over, we gave them a new one to fill out. We informed participants that they could terminate the experiment at any time, and that they should only divulge information that they were comfortable sharing. We also mentioned that we may publish anonymized quotes from their diaries. 

The booklet contained 25 pages and each page included the questions listed below. If participants were not able to find an appropriate answer, they provided an explanation. We asked the participants to write down these details as soon as possible after they performed a search. 

These were the questions on each page of the diary that participants answered as soon as they performed a search.
\begin{tight_enumerate}
            \item Date
            \item Time
            \item Duration of search task 
            \item What app or website did you access
            \item What were you searching for?
            \item Did you find what you were searching for at all? YES/NO
            \item If you did find your information, please continue by filling in the blanks with numbers: I performed \_ searches to find my information. I followed \_ links after leaving the search results page.
            \item Rate the difficulty of finding your information from 1--5 with 5 being very difficult. Add text to explain if you like.
\end{tight_enumerate}



\subsection{Results}

During the course of the diary study, we collected $868$ search entries with an average of 27 entries per person. 9\% of searches (33 out of 868) failed, not providing users with the information they sought. About a third (279 out of 868 search entries) were difficult (opposed to `easy' described below). 41.2\% used Google app, 49.4\% used a specific url, and the rest used other apps (i.e Amazon app). Participants performed an average of $1.2$ searches ($median=1, min =1, max=5$) and followed $2.5$ links ($median=2, min=0, max=39$) to find their information. Participants evaluated the task difficulty at an average of $1.9$ ($median=2,  stdev=1, min=1, max=5$) based on a Likert scale with 1 being very easy and 5 being very difficult. 

The search queries were analyzed from 2 different perspectives. First, whether the queries were imprecise or not. We define imprecise queries to consist of 2 or more intermediate queries or 3 or more links clicked. Second, whether the queries were difficult or easy. Too hard indicates failed, with a difficulty rating of $4$ or $5$, or more than $2$ minutes of work. Otherwise, the query was classified to be easy. 

Using these two classifiers, were were able to bin the queries into $4$ categories: We found that 49 \% of the queries were not imprecise and easy, 7 \% were not imprecise and difficult, 24 \% were imprecise and difficult, and 20 \% were imprecise and easy. Table \ref{} shows t

Although search was usually successful, it was difficult about a third of the time, especially when search was more open and exploratory. Even among easy searches, over half were near misses requiring several clicks to find the needed information. In sum, we believe the large majority of searches could have benefited from a tool that helped users navigate through the information neighborhoods typical of open and near miss searches. 

% Vidya: I don't think these two paras are relevant to this paper.

%The average time taken to record a search activity was $131.6$ seconds ($median=120, max=1200$). 
%People use applications more than websites for certain activities. On an average, 18.5\% (160 out of 868) of the searches were using applications than web portals. The most popular category searched using an application was \emph{shopping} (47\%). Surprisingly, participants frequently knew the exact keyword to put in as a search phrase. People use mobile apps for regular, standard search activities (i.e. weather, shopping, location, names, exchange rate, bus route, definition). For less frequent, more complicated searches, they will try it on a web portal but easily give up because of the screen size restriction and choose another method (i.e. go to their desktop, ask friends).

%
%We looked for duplicate themes from collected diary entries and did not start with a set of fixed categories. We adopted most of the categories from \cite{chi2008} and added new categories such as `\emph{how-to}', `\emph{unit conversion}', `\emph{definition}', and `\emph{name}'. There are 17 categories based on the diary entries.
%The \emph{how-to} category includes information about practical advice and detailed instruction of an activity. The \emph{unit conversion} includes kilometer to miles conversion, fahrenheit to celsius, and exchange rate. The \emph{definition} includes any need of terminology definition and detailed explanation. The \emph{name} includes search of a certain person or movie title. The largest category of collected entries was \emph{trivia} (51\%). They are the random thoughts from the participants such as ``story of shutter island". The second highest was \emph{shopping} (13\%), followed by \emph{point of interest} (10\%) and \emph{definition} (6\%).



%Vidya: I think this figure is hard to understand and looks too complicated.
%Figure \ref{fig:searchtype} categorizes search type of diary entries by success/failure, easy/difficult, open/specific queries, and hit or miss. Overall, the large blue section shows that people are mostly successful in finding information with their mobile device (though they may never attempt many challenging searches). About a third of searches are still difficult, and over half of difficult searches are open. 
%
%
%\begin{figure}[ht]
%\centering
%\includegraphics[width=3in]{images/searchtype}
%\caption{Search type by success/failure, easy/difficult, open/specific, and hit/miss. Blue: success, red: failure, saturated color: easy, bright color: difficult, empty dot: open, filled dot: specific, smaller dot: miss, bigger dot: hit}
%\label{fig:searchtype}
%\end{figure}



\subsection{Discussion and Design Implications for Imprecise Search}




\section{The GraphTiles interface}
We designed \textit{GraphTiles} to help mobile search users handle three challenging search types: open, exploratory search, where information foraging behavior is opportunistic and direction is weak at best; ill-formed search, where direction is known but difficult to articulate; and near miss searches, where the direction can be articulated, but the search engine leaves users some distance away from desired information. All of these search types would benefit from an interface that displays some of the local information neighborhood, allowing users to approach their information more directly. 
We designed \textit{GraphTiles} to be such an interface. 

With \textit{GraphTiles} (Figure \ref{fig:teaser} (a),(c)), we assume that users will employ search to find a locality of concern around a central node (e.g. for IMDb, ``near John Wayne"). As discussed above, position in the layout reflects link distance from the center. When necessary, users can drag a non-central node to the left to change the central node.We display links largely implicitly: every node in the middle column has an implied link to the central node, and every node in the right column is reachable from the middle column. To represent links between the middle and right columns we support both explicit link display and interactive reordering around a selected link. Explicit links appear only when both linked nodes are currently displayed. Reordering has the added benefit of accelerating access to off-screen nodes.

We considered a circular (or rectangular) layout to make better use of the blank space in the left column, with a scroll around the central node rather than along it, but discarded it so that we could provide a glimpse of a larger two-link neighborhood. A circular layout with a two-link neighborhood would require much smaller nodes (difficult to touch with a finger tip), and would fit poorly in rectangular mobile displays.

IMDb is largely a bipartite graph, with movies linking to crew members and vice versa. \textit{GraphTiles} quite appropriately exploits this in its interface, putting movies in one column, crew in the next, and assuming that there are no links between nodes in the same column. When graphs are not bipartite, representing within-column links explicitly is awkward at best. We generalize \textit{GraphTiles} to all graphs by relying on interactive reordering alone, and demonstrate this with a second database, the Seattle Band Map.


\section{Experiment: Comparison to IMDb's mobile site}

As a summative evaluation, we compared \textit{GraphTiles} to IMDb's mobile site. In \textit{GraphTiles}, we used both interactive reordering and the best explicit link representation, ---connecting nodes with lines ---a result from our other experiment (below). 

Our experiment focused on helping users answer ill-formed queries. In the context of IMDb, users often want to recommend a movie to a friend, but cannot remember the name of that movie, nor the name of any actors in that movie. They do however know that one of the actors in the movie was also in a movie they know. They move from movie to actor to movie. Movie buffs also often try their hand at casting future films. They cannot remember the actor's name, nor the name of the movie they remember them from. But, they do remember the name of a second actor in that movie. Thus, they move from actor to movie to actor.
We focused on answering imprecise queries of the same type. (We assumed that IMDb's mobile web app is a better solution for more precise queries such as ``The movies that John Wayne has acted in"). 

Figure \ref{fig:teaser} shows a comparison of the visuals used in \textit{GraphTiles} and IMDb's mobile website to answer the movie-person-movie(MPM) and person-movie-person(PMP) \textit{QueryTypes}. We expected that \textit{GraphTiles} would allow users to find answers to imprecise queries more quickly than IMDb's web app.




\subsection{Methods}

We compared \textit{GraphTiles} to IMDb's mobile site in an experiment with twenty participants, all of them employees at a large corporate research center. Each participant performed 120 information seeking tasks, using the same graph neighborhoods we used in our first experiment.

We used a fully crossed within subjects $2 \times 2$ design. As participants performed the tasks, we systematically altered two variables. \textit{Interface}, or the tool used to access the IMDb information, had two levels: \textit{GraphTiles} and the IMDb web app. \textit{QueryType} had two levels: a movie-person-movie (MPM) query or a person-movie-person (PMP) query. 
 If \textit{QueryType} was MPM, we asked participants to find the person who worked in two given movies. In this case, the central node at the left of the visualization was always a movie. If \textit{QueryType} was PMP, we asked participants to find the movie on which two given people collaborated. In this case, the central node at the left of the visualization was always a person. To answer the question, participants used a phone to scroll in the right column to find the second person's node, and then scroll in the middle column to find the movie connecting the two people, and select it.

In addition to displaying link lines, \textit{GraphTiles} here implemented interactive reordering, which highlights nodes' links to a selected node and moves them onscreen. Every participant performed 30 trials with each of the $2 \times 2 = 4$ experimental treatments. We grouped trials by \textit{Interface} into two blocks of 60 trials each. Thus participants performed all trials with the current Interface before moving on to the next. To combat the effects of fatigue and learning, we used complete counterbalancing across participants: half of them performed the \textit{GraphTiles} block first, the other half the web app block first. Within each of these blocks, we randomly ordered the levels of \textit{QueryType}. We randomized the order of graph neighborhoods without replacement, so that each participant saw each neighborhood exactly once.

\begin{figure*}[t]
\centering
\subfigure[Band and artist relationship]{\includegraphics[width=1.4in]{images/bab}}
\subfigure[Band relationship]{\includegraphics[width=1.4in]{images/bbb}}
\subfigure[Interactive reordering with dimmed image when not related.]{\includegraphics[width=1.4in]{images/fartz_reordering}}
\caption{Applying \textit{GraphTiles} to Seattle's music band data.}
\label{fig:musicband}
\end{figure*}

\subsubsection{Apparatus}

We implemented our interface on three Samsung SGH-i917 phones running Windows Phone 7.5, with an AMOLED display and a full capacitive touch screen. The monitor used to display questions was a $1920 \times 1200$ pixel Dell 24''. Participants interacted with the visualization on a phone by scrolling with a swipe gesture or selecting nodes with a long tap.

We obtained our IMDb graph neighborhoods using the official IMDb API, obtaining a large cross section of its database (approximately 3GB in size). We then randomly selected 60 nodes within the IMDb graph describing well known actors (supporting PMP queries), and 60 nodes describing well known movies (supporting MPM queries). We then sampled the two-link neighborhood around each actor (PMP) node by adding the top movies linked to it as indicated by IMDb's own API call; and then for each of those top movies, adding its top actors, again as indicated by IMDb's API call. We created two-link neighborhoods around movie (MPM) nodes similarly. The number of top movies returned by IMDb's API was generally much lower than the number of top actors. 


\subsection{Results and discussion}

All participants performed all trials correctly, so we report only completion times here. We tested significance using a two-factor repeated measures ANOVA. Only the two single variable effects were significant; they did not interact. 

When using \textit{GraphTiles}, participants were significantly ($F(1,19)=2291.833$, $p<0.001$) faster than when using the IMDb web app. Average completion time with \textit{GraphTiles} was 18.2s (SD 5.27), while with IMDb web app, it was 31.5s (SD 5.26).

Although its effect was significant ($F(1,19)=11.27$, $p<0.005$), \textit{QueryType}'s effect was not meaningful. The difference in completion times when participants looked for movies rather than persons was 0.6s: (25.0s for movies, 24.4s for persons). 


Results in fact exceeded our expectations, with \textit{GraphTiles} users almost twice as fast as IMDb web app users. \textit{GraphTiles} was designed for imprecise queries; IMDb probably was not. What remains to be seen is whether or not a single interface can support both precise and imprecise queries well.


\section{GraphTiles in non-bipartite graphs}

While real and practical, the IMDb graph is bipartite: nodes contain either people (e.g. actors) or movies. \textit{GraphTiles} quite appropriately exploits this structure, placing people and movies in different columns. However if \textit{GraphTiles} is to find use with more general applications, it must be tested with non-bipartite graphs. 

With this goal in mind, we used \textit{GraphTiles} to the Seattle Band Map ~\cite{seattleband}. In this database, music bands from the Pacific Northwest are linked if they share band members or have collaborated with one another. By preprocessing the database, we could create a bipartite graph of musicians and bands (Figure \ref{fig:musicband}(a)), but that is not our purpose here. 

Figure \ref{fig:musicband}(b) shows a non-bipartite band-band layout using lines to represent links. The challenge here is representing links that start and end within the same \textit{GraphTiles} column, which do not exist in bipartite graphs. Lines and most of the other explicit link representations we discussed perform poorly in such cases, since they are only displayed when both endpoints are onscreen, which will happen only rarely within the same column.

We believe interactive reordering is the best solution to this problem. In Figure \ref{fig:musicband}(c), the user selects the band `The Fartz', bringing all related bands onscreen or nearly so. Unrelated bands are dimmed out in the interface to further accentuate band connections.


\section{Experiment: Comparison of explicit link representations}


We considered how to display links explicitly on the mobile screen. It might be tempting simply to draw lines between linked nodes (Figure \ref{fig:linkrep}c), but \textit{GraphTiles} has unique characteristics that could make this solution untenable. As users scroll, nodes appear and disappear, meaning that linking lines do as well. Scrolling also causes the lines to move when they are onscreen, occluding a variety of other nodes and dynamically relocating link crossings (making a well-known drawback of link lines still worse). All of this dynamic behavior does not exist in most graph visualizations and could be quite disorienting during mobile search. .

In creating alternative designs for displaying explicit links, we were (loosely) inspired by the grouping principles of Gestalt psychology ~\cite{RefWorks:562}. The \textit{proximity} principle places nearby items in the same group. Because we could not use proximity alone to display complex many-to-many relationships, we approximated proximity with an iconic representation of the neighboring column (Figure \ref{fig:linkrep}(d)). Rectangles in the representation indicate the presence of links to the node in the same position in the neighboring column. \textit{Similarity} groups items that have similar properties such as color or texture (Figure \ref{fig:linkrep}(b) and \ref{fig:linkrep}(e)). Here, nodes containing the outline color or a thumbnail of a neighboring node are linked to that node. Like link lines (which use the Gestalt principle of \textit{connectedness}), all of these representations must dynamically change as the user scrolls and nodes move, but the changes are much more restrained.

\begin{figure*}[htb!]
\centering
\includegraphics[width=7in]{images/linkrep}
\caption{Different explicit link representations for \textit{GraphTiles}. (a) \emph{text:} nodes with the same name are linked. (b) \emph{color:} nodes with the same color are linked. (c) \emph{connectedness:} nodes with lines between them are linked. (d) \emph{proximity:} nodes at/containing the same vertical position are linked. (e) \emph{texture:} nodes with/containing the same image are linked.}
\label{fig:linkrep}
\end{figure*}



\subsection{Methods}

Using IMDb as a testbed, we compared connectedness-inspired lines to our alternative designs in a controlled experiment, and included a text-based link display (Figure \ref{fig:linkrep}(a)) as a control condition. In this condition, nodes with the same text were linked. Note that because we were testing only explicit link representations, we did not enable interactive reordering in this experiment.

We expected that connectedness-, color- and texture-inspired links would perform better than text-based or proximity-inspired links. Because of the unique dynamic qualities of the \textit{GraphTiles} visualization, we did not attempt to predict which link representation would be best.

\subsubsection{Design}

We used a fully crossed within subjects $5 \times 2 \times 2$ design. Link \textit{Depiction} had five levels: text-based as well as proximity-, color-, texture-, and connectedness-inspired representations. \textit{QueryType}, or the type of question asked, had two levels: a movie-person-movie (MPM) query or a person-movie-person (PMP) query. \textit{Size}, or the rough size of the surrounding graph neighborhood, had two levels: small or below median, and large or above median.


\subsubsection{Participants and procedure}

We had ten participants, all university students with normal or corrected-to-normal vision. We obtained informed consent from the participants, and asked them to read the instructions for the experiment. We then familiarized them with the task and link depictions using 10 training datasets, one for each combination of link \textit{Depiction} and \textit{QueryType}. Participants were free to ask verbal questions during training.

Participants then each performed $120$ information seeking tasks, each using a different graph neighborhood in the IMDb database, with median size of 115 nodes. On average, they completed all their tasks in one hour. Every participant performed six trials with each of the $5 \times 2 \times 2 = 20$ experimental treatments. We formed five blocks of $24$ trials each, each block corresponding to one \textit{Depiction}. Thus participants performed all trials with the current \textit{Depiction} before moving on to the next. To combat the effects of fatigue and learning, we sampled all the orderings of \textit{Depiction} using a $5 \times 5$ Latin Square. Within each of these \textit{Depiction} blocks, we formed two 12-trial \textit{QueryType} blocks. Half of the participants performed MPM questions first, half performed PMP questions first. Within each \textit{QueryType} block, participants performed 6 trials with small neighborhoods and 6 with large neighborhoods. We randomized the order of these trials. To avoid any confound between treatments and graph neighborhoods, we randomized the match of graph to treatment. Each participant saw each neighborhood only once.

For each task, participants answered a question displayed on a nearby monitor. 
As for the \textit{QueryType}, we used the same method as the previous summative experiment.
We recorded the time to complete each trial, and whether or not the participant performed the trial correctly. Participants were paid \$10 for their effort.



\subsection{Results}
\begin{figure}[ht]
\centering
\includegraphics[width=3in]{images/depictiongraph}
\caption{Average task completion times per depiction for the first experiment.}
\label{fig:experiment1}
\end{figure}

All participants completed all trials correctly, so we report only on completion time here. On completion times, we performed a single, three-factor repeated measures ANOVA. All single variable effects were significant. 

The connectedness-inspired link Depiction indeed supported the fastest information seeking performance ($F(4,36)=4.942, p<0.005$). Average completion times in seconds for each Depiction were: connectedness $10.1$s, texture $12.5$s, color $13.0$s, proximity $13.6$s, and text $15.7$s. We show the same times in Figure \ref{fig:experiment1}, along with standard error. Despite their drawbacks, link lines also have strengths: they are familiar to most viewers;  and they are simple, introducing only one primitive per link, while other representations require changes at both linked nodes.


Participants were much faster when asked to locate a person (the MPM \textit{QueryType}) than when asked to locate a movie (PMP) ($F(1,9)=43.869, p<0.001$). Average completion times for person queries were 10.5s, and for movies 15.5s. This is likely an effect of graph size rather than some more subtle task difference. Recall that IMDb's API returned many more top actors working on a movie than top movies in which an actor worked. This meant that PMP neighborhoods contained many more nodes than MPM neighborhoods.

Participants were faster when working with small graph Sizes than with large graph Sizes ($F(1,9)=83.911, p<0.001$). Average completion times for small graphs were $11.7$s, while for large graphs they were $14.2$s.

The only significant interaction occurred between the \textit{QueryType} and \textit{Size} variables ($F(1,9)=25.824, p=0.001$). When participants were asked to find movies in PMP neighborhoods, increasing graph Size had a large effect on completion times ($13.4$s vs. $17.6$s). When they were asked to find persons in MPM neighborhoods, \textit{Size}'s effect was minor ($10.1$s vs. $10.9$s). In PMP neighborhoods, graphs were larger, so increasing \textit{Size} had a larger effect.

Readers may wonder why average times in this experiment with \textit{GraphTiles} were lesser than they were in our first experiment. One cause may be the increased practice with \textit{GraphTiles} (10 training datasets) in this experiment.


\subsection{Discussion}
Results largely matched our expectations, with text-based and proximity-inspired links performing worst, texture- and color- inspired link \textit{Depictions} performing better, and connectedness-inspired link lines performing best. However, users were only about 20\% faster with link lines than with texture-inspired links containing thumbnails.



\section{Conclusion and Future Work}

As mobile devices become the dominant form of computing, mobile search will become increasingly important. In this paper we described \textit{GraphTiles}, a new search interface specifically designed to support open, imprecise, and near miss mobile queries. In an experimental evaluation, accessing the IMDb graph with \textit{GraphTiles} was nearly twice as fast as with the existing IMDb mobile web app.

\textit{GraphTiles} could use design improvements to maintain visual continuity. When users change the central node, they can quickly become disoriented. Future experiments might study how well \textit{GraphTiles} supports \textit{both} precise and imprecise queries, open and near miss search, as well as non-bipartite graphs. Our current implementation is quite visual. \textit{GraphTiles} will need improvement for largely textual search domains. 

We also plan to evaluate \textit{GraphTiles} on other devices such as tablets, where we might display larger neighborhoods. The comparative merits of each of our explicit link display techniques might be different when many more links must be displayed at the same time.

%\begin{table}[h]
%\centering
%\begin{center} {\footnotesize
%{\renewcommand{\arraystretch}{1.2}%
%\begin{tabular}{|l||l|}
%\hline
%\textbf{independent variable}  &  \textbf{ANOVA of time} \\
%\hline
% &     F(,)=, p= \\
%\hline
%\end{tabular} }} \quad
%\end{center}
%\caption{\footnotesize Significant main effects on time.}
%\label{table:timeAnova}
%\end{table}
%\balance

% Balancing columns in a ref list is a bit of a pain because you
% either use a hack like flushend or balance, or manually insert
% a column break.  http://www.tex.ac.uk/cgi-bin/texfaq2html?label=balance
% multicols doesn't work because we're already in two-column mode,
% and flushend isn't awesome, so I choose balance.  See this
% for more info: http://cs.brown.edu/system/software/latex/doc/balance.pdf
%
% Note that in a perfect world balance wants to be in the first
% column of the last page.
%
% If balance doesn't work for you, you can remove that and
% hard-code a column break into the bbl file right before you
% submit:
%
% http://stackoverflow.com/questions/2149854/how-to-manually-equalize-columns-
% in-an-ieee-paper-if-using-bibtex
%
% Or, just remove \balance and give up on balancing the last page.
%
%\balance

\bibliographystyle{acm-sigchi}
\bibliography{sample}
\end{document}
