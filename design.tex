

Before designing \textit{GraphTiles}, we carefully considered the constraints of mobile usage. First, mobile devices are little enough to fit in users' pockets, so visual features on their displays will be harder to see than they would be on desktop displays. Add in glare, instability, distraction and grime, and it is clear that mobile visualizations should \textit{display information efficiently, and filter to maintain clarity}. Second, mobile input is dominated by touch, and given small mobile displays, navigating and filtering interactions are crucial. This makes mobile visualization still more difficult, since information display must not only be clear and succinct, but also \textit{afford touch interaction}, requiring interface elements much larger than mice do. Third, given all the interaction mobile users need to see their data, mobile visualization must \textit{maintain visual continuity}.


Supporting intensive data analysis on mobile devices is unrealistic. But we believe that well-designed mobile visualizations can not only be an entertaining medium for casual browsing, but also help users answer their (in technological terms at least) ill-formed queries. Mobile visualizations should \textit{support casual data exploration}, and \textit{answer user questions by displaying local data neighborhoods}.


Information seeking in stationary or familiar settings occurred much more often than previously, and users frequently opted for interfaces specialized (e.g. native or web apps) to the information being sought. Mobile search systems should therefore \textit{enable answering specific questions quickly}, and \textit{support largely undirected, casual data exploration}. Supporting intensive data analysis may be unrealistic.


Mobile devices are little enough to fit in users' pockets or backpacks easily, and their displays correspondingly small. While today's mobile displays often contain just as many pixels as desktop displays, the fact remains that visual features on these displays will be smaller and therefore harder to see than they would be on desktop displays. Add in the compounding factors of glare, instability, distraction and grime, and it is safe to say that mobile visualizations should \textit{use display space very carefully}. Information should be displayed as efficiently as possible, and be filtered aggressively to maintain clarity. 

Mobile input is currently dominated by touch. If mobile users must interact with visualizations, the interface must provide largely error-free opportunities for navigation, zooming and requests for additional detail. Providing such opportunities with a touch interface makes mobile visualization still more challenging, since it must not only display information selectively and efficiently, it must be perceived to \textit{afford touch interaction}, which requires interface elements much larger than mice do on desktop displays.

Given these constraints, visualizations will very often be unable to display the entirety of their data, and requests for navigational actions such as panning and zooming will dominate interaction. Again, given the limited size of mobile displays and the imprecision of touch input, users can quite easily become disoriented. Visualizations on mobile devices should therefore be carefully designed to \textit{maintain visual context} by ensuring some visual continuity during navigational interaction.

Perhaps fortunately given all these constraints, users recognize that obtaining information with mobile devices is often more difficult than with desktop machines, and have adapted their use of mobile devices accordingly.


