The diary study shows that there is a spectrum of mobile search, ranging from general search to more specific precise search, and imprecise search falling somewhere between. Imprecise search is characterized by user difficulty describing the information being sought. This results in users reformulating queries, and spending extended time navigating through results from poorly described searches. While several apps and mobile websites exist to address the first two classes of mobile search~\cite{Cui:2008,Church:2011}, we designed \textit{GraphTiles} to help mobile search users handle the most challenging use case identified by our diary study, \textit{i.e.} imprecise search.

We sought to address these problems by: 
\begin{tight_itemize}
\item presenting a succinct overview of the results, given mobile constraints. This helps users find their information more quickly, and may help them recall more detail about the information they seek.
\item  enabling rapid navigation through the results with simple gestures, including scrolling and faceted search with the results themselves as parameters. Not only is this necessary given mobile constraints, it again helps users find information more quickly, and also helps avoid the necessity of new queries by drilling down in the results themselves. Our interaction design was influenced by Schneiderman�s information seeking mantra of detail-on-demand using overview with zoom and filter~\cite{Shneiderman:1996}.
\end{tight_itemize}
%
% With general search, users browse and cast a wide net. We expected a tool providing a good overview of search results to help them find the most interesting results. With imprecise search, users have difficulty characterizing their search, typically reformulating their queries and exploring intermediate results to hone in on and find reminders of what they seek. Again we expected a visual overview to aid users. 

The \textit{GraphTiles} interface displays a portion of the local entity-relationship neighborhood: a thumbnail of the current page alone in the left column, some pages one link away in the middle column, and other pages two links away in the right column. To see the complete neighborhood, users can scroll the central and right columns vertically. While this layout implies many links, it does not indicate exactly where the links are between the second and third columns. Users can reveal these locations by selecting a page thumbnail from these columns, triggering an interactive reordering that highlights pages linked to the selection and places them onscreen or nearly so. Users can restore the original (better known) order by deselecting the page. 

As shown in Figures \ref{fig:teaser}(a) and \ref{fig:teaser}(c), we assume that users will employ search to find a locality of concern around a central node (e.g. for IMDb, ``near John Wayne"), represented by a thumbnail alone in the left column. Distance in the \textit{GraphTiles} layout from this central node reflects relational distance from the center (e.g. for IMDb, degrees of working separation from John Wayne), with the middle column one link away, and the right column two links away. To see the complete two link neighborhood, users can scroll the central and right columns vertically. We display links largely implicitly: every node in the middle column has an implied link to the central node, and every node in the right column is reachable from the middle column. To represent links between the middle and right columns we support both explicit link display, and interactive reordering. Explicit links appear only when both linked nodes are currently displayed. With reordering, when users select a thumbnail from these columns, \textit{GraphTiles} highlights thumbnails linked to the selection and reorders to place them onscreen or nearly so. Users can restore the previous order by deslected the thumbnail. When necessary, users can drag a non-central node to the left to change the central node.

We considered a circular (or rectangular) layout to make better use of the blank space in the left column, with a scroll around the central node rather than along it, but discarded it so that we could provide a glimpse of a larger two-link neighborhood. A circular layout with a two-link neighborhood would require much smaller nodes (difficult to touch with a finger tip), and would fit poorly in rectangular mobile displays. Representing within column links explicitly can be confusing, so for such cases we rely on interactive reordering alone (see (c) in Figure \ref{fig:musicband}, depicting data from the Seattle Band Map ~\cite{seattleband}).