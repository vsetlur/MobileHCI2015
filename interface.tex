We designed \textit{GraphTiles} to help mobile search users handle the most challenging use case identified by our diary study: general or imprecise search. With general search, users browse and cast a wide net. We expected a tool providing a good overview of search results to help them find the most interesting results. With imprecise search, users have difficulty characterizing their search, typically reformulating their queries and exploring intermediate results to hone in on and find reminders of what they seek. Again we expected a visual overview to aid users. GraphTiles exploits the structure available in many online datasets to produce this overview. 

With \textit{GraphTiles} (Figure \ref{fig:teaser} (a),(c)), we assume that users will employ search to find a locality of concern around a central node (e.g. for IMDb, ``near John Wayne"), represented by a thumbnail alone in the left column. Distance in the GraphTiles layout from this central node reflects relational distance from the center (e.g. for IMDb, degrees of working separation from John Wayne), with the middle column on link away, and the right column two links away. To see the complete two link neighborhood, users can scroll the central and right columns vertically. We display links largely implicitly: every node in the middle column has an implied link to the central node, and every node in the right column is reachable from the middle column. To represent links between the middle and right columns we support both explicit link display, and interactive reordering. Explicit links appear only when both linked nodes are currently displayed. With reordering, when users select a thumbnail from these columns, GraphTiles highlights thumbnails linked to the selection and reorders to place them onscreen or nearly so. Users can restore the previous order by deslected the thumbnail. When necessary, users can drag a non-central node to the left to change the central node.

We considered a circular (or rectangular) layout to make better use of the blank space in the left column, with a scroll around the central node rather than along it, but discarded it so that we could provide a glimpse of a larger two-link neighborhood. A circular layout with a two-link neighborhood would require much smaller nodes (difficult to touch with a finger tip), and would fit poorly in rectangular mobile displays. Representing within column links explicitly can be confusing, so for such cases we rely on interactive reordering alone (see (c) in Figure \ref{fig:musicband}, depicting data from the Seattle Band Map ~\cite{seattleband}).