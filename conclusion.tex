As mobile devices become the dominant form of computing, mobile search will become increasingly important. In this paper we described \textit{GraphTiles}, a new search interface specifically designed for imprecise search. In a diary study, general and imprecise search proved to be the focus of most user difficulty. In an experimental evaluation, accessing the IMDb graph for imprecise search with \textit{GraphTiles} was nearly twice as fast as with the existing IMDb mobile web app.

A number of possible design improvements to \textit{GraphTiles} could be studied in future work. The current design is optimized for smartphones; on devices such as tablets, \textit{GraphTiles} might display larger neighborhoods. \textit{GraphTiles} could also use improvements to maintain visual continuity when users change the central node: currently users can quickly become disoriented. 

Several limitations of and questions raised by our work also deserve follow-up. Most importantly, how easily could \textit{GraphTiles} be generalized?  There are two primary data constraints that we exploited in \textit{GraphTiles}. First, the presence of visual thumbnails, which are an effective way of improving experience and summarizing available information~\cite{Setlur:2011}. Second, the existence of a structuring graph of entities and their relationships, which enable users to navigate through information in an intuitive manner. In our experience, there are many sites that match these constraints, including IMDb~\cite{imdb}, AllMusic~\cite{allmusic} and Allrecipes~\cite{allrecipes}. 

When sites do not contain thumbnails, one might substitute text. For example, the \textit{GraphTiles} interface could initially display abridged ingredients of salmon dishes with additional interaction for showing longer descriptions. Another option might show summarizing thumbnails built on the fly, containing both text and imagery~\cite{Setlur:2011}. For sites without an entity-relationship graph, a navigable structure is still a necessity. It may be possible to use the links in a webpage or search engine results to provide this structure.

Will or should \textit{GraphTiles} always be a special case, or can it be part of a unified solution for specific as well as imprecise and general search? Researchers might examine this question by folding \textit{GraphTiles} into a more general information interface. Finally, it could be profitable to learn about the various contributions to mobile search difficulty in general vs. imprecise search. We were not able to disentangle the two in the diary study we used here, but future work might employ a different measurement method.
