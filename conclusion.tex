As mobile devices become the dominant form of computing, mobile search will become increasingly important. In this paper we described \textit{GraphTiles}, a new search interface specifically designed for general browsing and imprecise search. In a diary study, these more complex types of search proved to be the focus of most user difficulty. In an experimental evaluation, accessing the IMDb graph for imprecise search with \textit{GraphTiles} was nearly twice as fast as with the existing IMDb mobile web app.

A number of possible design improvements to \textit{GraphTiles} could be studied in future work. The current design is optimized for smartphones; on devices such as tablets \textit{GraphTiles} might display larger neighborhoods. \textit{GraphTiles} could also use improvements to maintain visual continuity when users change the central node: currently users can quickly become disoriented. 

Several limitations and open questions in our work also deserve followup. How easily could \textit{GraphTiles} be generalized? Our current implementation is quite visual, and assumes structured data sources. \textit{GraphTiles} will need improvement for more textual and less structured data. Will or should \textit{GraphTiles} always be a special case, or can it be part of a unified solution for specific as well as imprecise and general search? Finally, it could be profitable to learn about the various contributions to mobile search difficulty of general vs. imprecise search. We were not able to disentangle the two in the diary study we used here, but future work might employ a different measurement method.
