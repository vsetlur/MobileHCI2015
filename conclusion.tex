As mobile devices become the dominant form of computing, mobile search will become increasingly important. In this paper we described \textit{GraphTiles}, a new search interface specifically designed to general browsing and imprecise search. In a diary study, these types of search proved to be the focus of most user difficulty with mobile search. In an experimental evaluation, accessing the IMDb graph for imprecise search with \textit{GraphTiles} was nearly twice as fast as with the existing IMDb mobile web app.

\textit{GraphTiles} could use design improvements to maintain visual continuity. When users change the central node, they can quickly become disoriented. Our current implementation is quite visual and relies on thumbnail images. \textit{GraphTiles} will need improvement for more textual search domains. We also plan to evaluate \textit{GraphTiles} on other devices such as tablets, where we might display larger neighborhoods. 

Future experiments might study how well \textit{GraphTiles} supports specific \textit{as well as} imprecise and general queries.  

% Questions that may need discussing: how common is imprecise vs. general search? We don't say. How widely can our solution be generalized? It relies on both images and entity relationship structure. Why did users have particular challenges with the general and imprecise questions? Maybe the answer is obvious: it's more complicated. How might one interface support both specific and general/imprecise search? 